\section{Dataset}
%Describe here your dataset (dimension of data, number of images per class, number of classes, features representatives of each class). Add illustrative images of the elements in your dataset to help the reader understand what each class entails.

	The first dataset consists of a portrait of one person. In some portraits, the person was worn glasses. Thus, we divided the dataset into two different classes: portraits with (figure \ref{fig:figure1_2}) and without (figure \ref{fig:figure1_1}) glasses (30 samples per class). Images were taken using an iPhone 6S located directly in front of the face and they had a resolution of 1932x1932. Therefore each image has more than 3 million dimensions. Additionally, pictures were taken with the same background in order to set a reference for all the dataset.
\begin{figure}[!hp]
\centering
	\begin{subfigure}[t]{0.18\textwidth}
    \centering
	\includegraphics[height=0.05\textheight]{dataset/IMG_0992.jpg}
	\captionsetup{font=small}
	\caption{\bf Class 1 : Portrait without glasses}
	\label{fig:figure1_1}
	\end{subfigure}
    \begin{subfigure}[t]{0.18\textwidth}
    \centering
    \includegraphics[height=0.05\textheight]{dataset/IMG_1077.jpg}
	\captionsetup{font=small}
	\caption{\bf Class 2 : Portrait with glasses}
	\label{fig:figure1_2}
    \end{subfigure} 
    \begin{subfigure}[t]{0.18\textwidth}
    \centering
	\includegraphics[height=0.05\textheight]{dataset/IMG_1012.jpg}
	\captionsetup{font=small}
	\caption{\bf subclass a : Happy}
    \label{fig:figure1_3}
	\end{subfigure}
    \begin{subfigure}[t]{0.18\textwidth}
    \centering
    \includegraphics[height=0.05\textheight]{dataset/IMG_0997.jpg}
	\captionsetup{font=small}
	\caption{\bf subclass b : Neutral}
    \label{fig:figure1_4}
    \end{subfigure}
    \begin{subfigure}[t]{0.18\textwidth}
    \centering
    \includegraphics[height=0.05\textheight]{dataset/IMG_1452.jpg}
	\captionsetup{font=small}
	\caption{\bf subclass c : Sad}
    \label{fig:figure1_5}
    \end{subfigure}
\caption{Image examples of each class and subclass}
\end{figure}
    The dataset also contains 3 subclasses which described the level of happiness of the person as shown in figure \ref{fig:figure1_3}, \ref{fig:figure1_4}, \ref{fig:figure1_5}. Thus, the second dataset was made based on these subclasses. 20 samples were taken for each subclass. The scale of happiness was in the range of [0:1]. 0 corresponds to a very sad humor, whereas 1 very happy. This dataset will be used for the section \ref{regression}.
    

